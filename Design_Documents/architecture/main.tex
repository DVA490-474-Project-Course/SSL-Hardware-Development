\documentclass[a4paper,12pt]{article}

% Packages for formatting
\usepackage{geometry}
\usepackage{array}
\usepackage{lipsum}
\geometry{margin=1in}
\usepackage{booktabs}
% For better table formatting
\usepackage[table]{xcolor}
% For color
\usepackage{graphicx}
% For including images, if necessary
\usepackage{tabularx}
% For including images, if necessary
\usepackage{longtable}
\usepackage{amsmath}
% For mathematical symbols, if needed

% Custom command for better table spacing
\newcolumntype{L}[1]{>{\raggedright\arraybackslash}p{#1}}
\newcolumntype{C}[1]{>{\centering\arraybackslash}p{#1}}
\newcolumntype{R}[1]{>{\raggedleft\arraybackslash}p{#1}}

\title{Bill of Materials (BOM) Explanation}
\author{ Pontus Svensson / RoboCup }
\date{\today}

\begin{document}

  \maketitle

  \section*{Introduction} This document provides an explanation of the
  Bill of Materials (BOM) we want to use in the project. Each component
  listed in the BOM is described in detail, including its purpose and
  other relevant specifications.

  \section*{Component Overview} The table below outlines the components
  used in this project along with their purpose and additional
  information.
  \begin{centering}
    \begin{longtable}{|p{3cm}|p{3cm}|p{3cm}|p{1cm}|p{2cm}| }
      \caption{Component Descriptions for the Project}
      \label{tab:component_bom}
      \\ \hline \rowcolor{gray!50} \textbf{Component} &
      \textbf{Description} & \textbf{Purpose} & \textbf{\#} & \textbf{á
        price (price per robot) SEK}\\ \endhead \hline

      \rowcolor{gray!50} \textbf{Total price for 1 robot} &
      \textbf{8398.47} \endlastfoot \hline

      DF45L024048-A & Brushless direct current (BLDC) motor with
      integrated hall sensors for the wheels & Used to spin the wheels
      of the robot. & 4 & 830.4 (3273.60)\\ \hline Hobbywing FPV XRotor
      3110 900KV & Brushelss DC motor & High revolutions per minute
      (RPM) motor used to control the dribbler. & 1 & 175.20 (175.20) \\
      \hline B-G431B-ESC1 & BLDC motor driver & Motor driver with
      embedded $\mu\text{Controller}$ current sensing and hall sensing
      to form a closed-loop control algorithm & 5 & 208.96 (1044.8) \\
      \hline NUCLEO-H723ZG & $\mu\text{Controller}$ & Computational
      power and real-time processing capabilities, supports
      $\mu\text{ROS}$ & 1 & 322.58 (322.58) \\ \hline Raspberry Pi 4
      Model B/8GB & Single-board computer & Processing camera input and
      performing local path planning & 1 & 979 (979) \\ \hline
      SX1280IMLTRT & Radio frequency (RF) transceiver & Used to transmit
      data over 2.4Ghz network & 1 & 75.44 (75.44) \\ \hline SKY66122-11
      & Integrated front-end-moduel (FEM) & Simplified integration with
      the RF circuit & 1 & 40.48 (40.48) \\ \hline 6s 1300mAh -120C -
      GNB HV XT60 & LiPo-battery & Used to power the robot & 1 & 351.20
      (351.20) \\ \hline LT3750 & Charging controller for the capacitors
      of the kicker & Charge controller for the kicker circuit & 1 &
      146.93 (146.93)\\ \hline iC-PX2604 + PX01S 26-30 & Wheel encoders
      & Will be used for odometry of the robot & 4 & 224.40 (897.60) \\
      \hline WSEN-ISDS 6 Axis IMU & 6-DoF IMU & Well be used for
      odometry of the robot & 10 & N/A\\ \hline Raspberry Pi Kameramodul
      3 & Camera & Provide images in front of the robot to detect the
      ball and obstacles & 1 & 369 (369) \\ \hline IR Break Beam Sensor
      - 5mm LEDs & Infrared (IR) sensor & Used to detect if the ball is
      close to the robot & 1 & 99 (99) \\ \hline JST 6B-PH-K-S &
      Connector & Hall sensor connector from the motor & 4 & 3.85 (15.4)
      \\ \hline JST B5P-VH & Connector & Motor connector & 4 & 4.06
      (16.24) \\ \hline Connectors & Passive component & Supplied by
      Würth & N/A & N/A \\ \hline Shaft hub with clamping bracket 4mm &
      Coupler & Couple the wheels with the motor shaft & 4 & 139 (556)
      \\ \hline Bearings & Bearings & Make the roller spin (dribbler) &
      2 & 18 (36)\\ \hline Resistors & Passive component & Supplied by
      Würth or 326 & N/A & N/A \\ \hline Capacitors & Passive component
      & Supplied by Würth or 326 & N/A & N/A \\ \hline Voltage
      regulators & DC/DC buck converters & Supplied by Würth & N/A & N/A
      \\ \hline Solenoid & Solenoid & Supplied by MDU & 1 & N/A \\
      \hline PCB & Printed circuit board (PCB) & The students will
      supply any custom PCB designed & 2 & N/A\\ \hline

    \end{longtable}
  \end{centering}

  \section*{Reason for component choice}

  \subsection*{DF45L024048-A} The BLDC motor has a high torque at low
  speeds and can reach high revs per minute (RPM). This is necessary to
  ensure fast acceleration at low speeds. In order to have precise
  control of the motors, positional feedback of the magnets position and
  the phase current can be used to form a closed-loop PID system. The
  $\mu\text{Controller}$ would only need to send the desired velocity to
  the ESC and the closed-loop control will make sure the desired speed
  is met. On the contrary, using a BLDC motor without sensor feedback,
  additional components would be required, for example external hall
  sensors and external current sensing iCs. This would add complexity
  and introduce integration issues. We also have to take the size of the
  motor into account. Having to large footprint on the motors would
  cause the kicker (solenoid) to not fit in the chassi. Sensored BLDC
  motors are mainly used for RC cars, where positional accuracy is
  required for low speeds and they have a larger footprint and cheaper
  alternatives compared to the DF45L024048-A did not exist.

  \subsection*{Hobbywing FPV XRotor 3110 900KV} The requirements for the
  dribbler motor is that it can reach high RPM (around $10\:000\text{
    RPM}$). Positional accuracy for the dribbler motor does not have to
  be as precise compared with the motor for the wheels. Since we have
  current feedback for each phase we can calculate the RPM of the motor.

  \subsection*{B-G431B-ESC1} The chosen electronics speed controller
  (ESC) has an STSPIN32F0A system in package chip which has an
  integrated STM32 with hall sensor decoding logic and current sensing
  capabilities. This makes this ESC a good fit with the DF45L024048-A
  BLDC motor. The PID system can be run on this chip for each motor
  offloading computational loads from the $\mu\text{Controller}$ and
  allow for precise movement and rapid acceleration which is critical to
  make quick directions changes and dribbling. The hall sensor feedback
  support enables accurate rotor position allowing for good torque
  control at low speeds. The size and weight of the ESC does also have
  to be taken in consideration, the B-G431B-ESC1 has a small footprint
  with a relatively low weight $286\text{g}$. With all the components
  integrated on one board it will make the assembly process easier and
  reduce any external components e.g. current sensing or hall sensing.
  The programming for the integrated STM32 is done using STM32 Motor
  Control Software Development Kit which is a graphical programming
  environtment from ST.
  \begin{longtable}{|p{3cm}|p{6cm}|p{6cm}|}
    \caption{Pros and Cons of the B-G431B-ESC1 Motor Driver in the Robot
      Setup} \\ \hline \textbf{Category} & \textbf{Pros} & \textbf{Cons
    } \\ \hline \endfirsthead

    \multicolumn{3}{c}{{\tablename\ \thetable{} -- continued from
        previous page}} \\ \hline \textbf{Category} & \textbf{Pros} &
    \textbf{Cons} \\ \hline \endhead

    \hline \multicolumn{3}{|r|}{{Continued on next page}} \\ \hline
    \endfoot

    \hline \endlastfoot

    \textbf{Integration and Control} & - Integrated microcontroller for
    closed-loop control. \newline - Built-in current sensing and Hall
    sensor feedback for precise, closed-loop control. \newline - Smooth
    low-speed operation ideal for motor control. & - Requires
    familiarity with STM32CubeIDE for firmware development, but
    extensive documentation and community support make learning
    manageable. \\ \hline \textbf{Compatibility with Motors} & - Direct
    support for Hall-effect sensors in BLDC motors, perfect for
    \textbf{DF45L024048-A} motors. \newline - Capable of handling
    high-speed applications, like the dribbler motor (\textbf{Hobbywing
      FPV XRotor 3110}). & - Limited to one motor per ESC, but its
    compact size allows for easy integration of multiple units in the
    robot. \\ \hline \textbf{Power Handling} & - Supports up to 48V,
    making it fully compatible with the \textbf{6s 1300mAh} LiPo
    battery. \newline & - Potential heat buildup under heavy loads, but
    simple heat sinks or airflow can mitigate this for sustained
    performance. \\ \hline \textbf{Size and Integration} & - Compact
    design simplifies installation in space-constrained robots. \newline
    - Reduces the need for additional wiring and external controllers,
    leading to cleaner and more reliable assembly. & - Using multiple
    ESCs requires careful layout, but it results in a modular design
  \end{longtable}

  \subsection*{NUCLEO-H723ZG} The NUCLEO-H723ZG $\mu\text{Controller}$
  is chosen because of its high computational abilities which is based
  of the STM32H7237G chip. The $\mu\text{Controller}$ will be required
  to collect the data from the sensors, perform calculations with the
  data, send the data to the team server.

  \subsection*{Raspberry Pi 4 Model B/8GB} \subsection*{SX1280IMLTRT +
    SKY66122-11} \subsection*{iC-PX2604 + PX01S 26-30}
  \subsection*{WSEN-ISDS 6 Axis IMU} \subsection*{Raspberry Pi
    Kameramodul 3} \subsection*{IR Break Beam Sensor - 5mm LEDs}
  \subsection*{Connectors} \subsection*{Shaft hub with clamping bracket
    4mm} \subsection*{Bearings} \subsection*{Resistors}
  \subsection*{Capacitors} \subsection*{Voltage regulators}
  \subsection*{Solenoid} \subsection*{PCB} \section*{Conclusion} This
  document serves as a reference for understanding the role of each
  component in the project. For any further technical details, please
  refer to the respective datasheets provided by the manufacturers.

\end{document}
